\documentclass[14pt]{extarticle}
\usepackage[left=15mm, top=15mm, right=20mm, bottom=15mm, nohead, footskip=10mm]{geometry}
\usepackage[T2A]{fontenc}
\usepackage[english, russian]{babel} % Языки: русский, английский
\usepackage[utf8]{inputenc}
\usepackage{graphicx}
\usepackage{amsmath}
\usepackage{animate}
\usepackage{color} %% это для отображения цвета в коде
\usepackage{listings} %% собственно, это и есть пакет listings
\usepackage{subcaption}
\usepackage{caption}
\DeclareCaptionFont{white}{\color{white}} %% это сделает текст заголовка белым
%% код ниже нарисует серую рамочку вокруг заголовка кода.
\DeclareCaptionFormat{listing}{\colorbox{gray}{\parbox{\textwidth}{#1#2#3}}}
\captionsetup[lstlisting]{format=listing,labelfont=white,textfont=white}

\lstset{ %
language=Java,                 % выбор языка для подсветки (здесь это С)
basicstyle=\small\sffamily, % размер и начертание шрифта для подсветки кода
numbers=left,               % где поставить нумерацию строк (слева\справа)
numberstyle=\tiny,           % размер шрифта для номеров строк
stepnumber=1,                   % размер шага между двумя номерами строк
numbersep=5pt,                % как далеко отстоят номера строк от подсвечиваемого кода
backgroundcolor=\color{white}, % цвет фона подсветки - используем \usepackage{color}
showspaces=false,            % показывать или нет пробелы специальными отступами
showstringspaces=false,      % показывать или нет пробелы в строках
showtabs=false,             % показывать или нет табуляцию в строках
frame=single,              % рисовать рамку вокруг кода
tabsize=2,                 % размер табуляции по умолчанию равен 2 пробелам
captionpos=t,              % позиция заголовка вверху [t] или внизу [b]
breaklines=true,           % автоматически переносить строки (да\нет)
breakatwhitespace=false, % переносить строки только если есть пробел
escapeinside={\%*}{*)}   % если нужно добавить комментарии в коде
}


\begin{document}

\begin{center}
\hfill \break
\large{Министерство науки и высшего образования Российской федерации}\\
\footnotesize{ФЕДЕРАЛЬНОЕ ГОСУДАРСТВЕННОЕ БЮДЖЕТНОЕ ОБРАЗОВАТЕЛЬНОЕ УЧРЕЖДЕНИЕ}\\
\footnotesize{ВЫСШЕГО ПРОФЕССИОНАЛЬНОГО ОБРАЗОВАНИЯ}\\
\small{\textbf{«АЛТАЙСКИЙ ГОСУДАРСТВЕННЫЙ УНИВЕРСИТЕТ»}}\\
\hfill \break
\normalsize{Институт цифровых технологий, электроники и физики}\\
 \hfill \break
\normalsize{Кафедра вычислительной техники и электроники}\\
\hfill\break
\hfill \break
\hfill \break
\hfill \break
\large{Курс <<Методы оптимизации>>\\ Отчет по лабораторной работе №1\\ <<Графический метод решения задач линейного программирования>>}\\
\end{center}
\hfill \break
\hfill \break
\hfill \break
\hfill \break
\hfill \break

\normalsize{
  \begin{flushright}
    \begin{tabular}{rcr}
      & Выполнил: & студент 585гр.\\\\
      & Роженцев А.К. &\underline{\hspace{3cm}}\\\\
      & Проверил: ст.пр.\\\\
      & Уланов П.Н. & \underline{\hspace{3cm}}
    \end{tabular}
  \end{flushright}
}
\hfill \break
\hfill \break
\hfill \break
\hfill \break
\hfill \break
\begin{center} Барнаул 2020 \end{center}
\thispagestyle{empty}
\newpage
\tableofcontents
\newpage
\section{Цель работы}
Изучить решение задач линейного программирования графическим способом.
\section{Теория}
Многоугольник решений – выпуклое множество, область допустимых решений задачи.\\
Линия уровня – прямая, уравнение которой получается из целевой функции, если ее приравнять постоянной величине.\\
Опорная прямая – линия уровня, имеющая общие точки с ОДР и расположенная
так, что ОДР целиков находится в одной из полуплоскостей.\\
Алгоритм решения задач линейного программирования:\\
Строится область допустимых значений
\begin{enumerate}
    \item  Строится вектор $\Vec{n}$= (${c_1},{c_2}$) с точкой приложения в начале координат.
    \item  Перпендикулярно вектору ��⃗ проводится одна из линий уровня.
    \item  Линии уровня перемещаются параллельно самой себе до положения
опорной прямой. На этой прямой находится максимум или минимум
функции.
\end{enumerate}

\section{Практическое задание }
Вариант 15.\\
Необходимо найти минимальное значение функции \[Z(x) = 8x_1 + 9x_2 \rightarrow min\] при ограничениях:

\begin{cases}
    \[13x_1 + 15x_2 \leq 93\\
        9x_1 + 7x_2 \geq 29\\
       -3x_1 + 4x_2 \geq -29\\
        x_1\geq 3,5\]
\end{cases}
\newpage

\section{Решение}
Изобразим на плоскости систему координат O${x_1,x_2}$ и построим граничные прямые ОДР:

\begin{equation}
    \begin{aligned}
    13x_1 + 15x_2 \leq 93\\
        &x_1 = 0\\
        &x_2 = 6.2\\
        &x_2 = 0\\
        &x_1 = 7.1528\\
    \end{aligned}
\end{equation}
\newline
\begin{equation}
    \begin{aligned}
    9x_1 + 7x_2 \geq 29\\
    &x_1 = 0\\
    &x_2 = 4.14286\\
    &x_2 = 0\\
    &x_1 = 3.2222\\
    \end{aligned}
\end{equation}
\newline

\begin{equation}
    \begin{aligned}
    -3x_1 + 4x_2 \geq -29\\
        &x_1 = 0\\
        &x_2 = -7.25\\
        &x_2 = 0\\
        &x_1 = 0.75\\
    \end{aligned}
\end{equation}



\newpage
\begin{figure}[!h]
  \centering
  \includegraphics[width=0.65\textwidth]{fin_111.png}
  \caption{Область допустимых значений}
\end{figure}

\newpage
Находим точку минимума, используя пересечения двух прямых удовлетворяющих условию, решаем систему:
\newline
\newline
\begin{cases}
   \[9x_1 + 7x_2 \geq 29\\
    -3x_1 + 4x_2 \geq -29\]
\end{cases}
\newline
\newline
Находим значения ${x_1}$ = 5.57425 и ${x_2}$ = -3.0211.\\
Найдем минимальное значение функции \[Z(x) = 8x_1 + 9x_2 \rightarrow min\] подставив ${x_1}$ = 5.57425 и ${x_2}$ = -3.0211

\[Z(x) = 8\cdot5.57425 + 9\cdot(-3.0211) = 17.4041\]



\newpage
\section{Построение в Gnuplot}

\begin{lstlisting}
set xr [0:10]
set yr [-10:10]
set size ratio 2 
set xtics 0,1,10
set ytics -10,1,10
set grid

plot 6.2-0.866667*x,
    4.14286 - 1.28571*x,
    -7.2 + 0.75*x,
    2.1 - 0.918706552*x,
    "1.txt" using 1:2:3:4 with vectors filled head lw 1,
    "2.txt" using 1:2:3:4 with vectors filled head lw 1,
    '-' w p ls 5 
    5.57425 -3.0211
    e
\end{lstlisting}

\section{Вывод}
Графический метод решения задач линейного программирования основан на геометрической интерпретации допустимых решений и целевой функции задачи. Наглядность этого метода достигается лишь на плоскости, поэтому графическое представление задачи возможно лишь в двумерном пространстве. Этот метод пригоден для решения задач ЛП с двумя переменными.
\end{document}
